\section{Predictions over time (ML)}

We want to predict future events.
\begin{itemize}
    \item Will the stocks of company $A$ increase in value today?
    \item Will it rain today?
\end{itemize}

For instance, you have some training data, that you split in, say, $n$ chunks.
You train, independently, a model on each chunk.
At testing time, you get the prediction of each of the $n$ models, and you have to select one of them.


\subsection{Expert model}
    We consider the expert model.
    We have $n$ "experts" that make (binary) predictions for future events. We take into consideration these predictions, and we then compare them with what actually happens.

    That is, for $i = 1, 2, \dots, T$, we are given a vector $(Y_{1t}, Y_{2t}, \dots, Y_{nt})$, where $Y_{it} \in \B$ is the prediction of expert $i$ for time $t$.

    We then have to "bet" on whether nature will, at time $t$, show us $0$ or $1$.
    Let $z_t \in \B$ be our prediction for time $t$.

    Nature shows us her bit $x_t \in \B$ for time $t$.

    We pay $1$€ is $z_t \neq x_t$.

    In Algorithm~\ref{alg:halving} a simple procedure for choosing experts' predictions.

    \begin{algorithm}
    \caption{Halving algorithm}\label{alg:halving}
    \begin{algorithmic}%[1]
            \State~$S \gets [n]$

            \For{$t = 1, \dots, T$}
                \State~$A_t \gets \cardinality{ \{ i \st i \in S \wedge Y_{it} = 1 \} }$
                \State~$B_t \gets \cardinality{ \{ i \st i \in S \wedge Y_{it} = 0 \} }$
                \If{$A_t \geq B_t$}
                    \State~$z_t \gets 1$
                \Else
                    \State~$z_t \gets 0$
                \EndIf
                \State~we "see" nature's $x_t$ (and pay $1$€ is $x_t \neq z_t$)
                \State~$S \gets S \setminus \{ i \st Y_{it} \neq x_t \}$
            \EndFor
    \end{algorithmic}
\end{algorithm}

    What Algorithm~\ref{alg:halving} does is it considers what the majority of the experts are predicting at time $t$, and all that are wrong will not be considered anymore.

    \begin{theorem}
        If there exists a perfect expert (that is, if $\exists i^* \in [n]$ s.t. $\forall t$, $Y_{i^*t} = x_t$), then the halving algorithm makes $m$ mistakes, with $m \leq \lfloor \log_2 n \rfloor$.
    \end{theorem}

    \begin{proof}[Proof (sketch)]
        If we make a mistake at time $t$, then at least $\frac{\cardinality{S}}{2}$ (currently trustworthy) experts make a mistake at time $t$.
        At the outset, $\cardinality{S} = n$.
    \end{proof}

    In Algorithm~\ref{alg:iter_halving} we consider a slightly modified version of Algorithm~\ref{alg:halving}, where, in case we've removed all the experts, we put them all back.

    \begin{algorithm}
    \caption{Iterative halving algorithm}\label{alg:iter_halving}
    \begin{algorithmic}%[1]
            \State~$S \gets [n]$

            \For{$t = 1, \dots, T$}
                \State~$A_t \gets \cardinality{ \{ i \st i \in S \wedge Y_{it} = 1 \} }$
                \State~$B_t \gets \cardinality{ \{ i \st i \in S \wedge Y_{it} = 0 \} }$
                \If{$A_t \geq B_t$}
                    \State~$z_t \gets 1$
                \Else
                    \State~$z_t \gets 0$
                \EndIf
                \State~we "see" nature's $x_t$ (and pay $1$€ is $x_t \neq z_t$)
                \State~$S \gets S \setminus \{ i \st Y_{it} \neq x_t \}$

                \If{$S = \emptyset$}
                    \State~$S \gets [n]$
                \EndIf
            \EndFor
    \end{algorithmic}
\end{algorithm}

    Let $m^* = m_T^*$ be the number of mistakes of the best expert in the $T$ rounds.

    \begin{theorem}
        The number of mistakes of Algorithm~\ref{alg:iter_halving} is
        \[ m \leq (m^* + 1)(\log_2 n + 1) \]
    \end{theorem}

    \begin{proof}[Proof (sketch)]
        We could cut the timespan in $m^* + 1$ pieces (cut at the mistakes of a best expert).
    \end{proof}

    In Algorithm~\ref{alg:weight_majority} we consider a different approach.
    Instead of simply counting how many $0$'s and $1$'s we get for the experts (removing them when they are wrong), we always consider all the experts, but weighting their answers based on how reliable they've been so far.
    At the outset, we assume every expert is perfectly reliable, and dimish their "reliability" each time they are wrong (that is, we give less weight to their answers).

    \begin{algorithm}
    \caption{Weighted majority}\label{alg:weight_majority}
    \begin{algorithmic}%[1]
            \State~assign a weight $w_i = 1$ to expert $i$, $\forall i \in [n]$

            \For{$t = 1, \dots, T$}
                \State~$A_t \gets \sum_{i : Y_{it}=1} w_i$
                \State~$B_t \gets \sum_{i : Y_{it}=0} w_i$

                \If{$A_t \geq B_t$}
                    \State~$z_t \gets 1$
                \Else
                    \State~$z_t \gets 0$
                \EndIf
                \State~we "see" nature's $x_t$ (and pay $1$€ is $x_t \neq z_t$)

                \For{$i = 1, \dots, n$}
                    \If{$Y_{it} \neq x_t$}
                        \State~$w_i \gets \frac{w_i}{2}$
                    \EndIf
                \EndFor
            \EndFor
    \end{algorithmic}
\end{algorithm}

    \begin{theorem}
        Algorithm~\ref{alg:weight_majority} makes number of mistakes
        \[ m \leq 2.41(m^* + \log_2 n) \]
    \end{theorem}

    \begin{proof}
        Let $w_i^t$ be the weight of expert $i$ at the end of round $t$ (after the weight-update step).
        Also, let $w_i^0 = 1$, $\forall i \in [n]$.

        We define a potential
        \[ W^t = \sum_{i=1}^{n} w_i^t \]

        A.
        \[ W^0 = \sum_{i=1}^{n} w_i^0 = \sum_{i=1}^{n} 1 = n \]

        B. $W^t = \sum_{i=1}^{n} w_i^t \geq \sum_{i=1}^{n} w_i^{t+1} = W^{t+1}$.
        Thus,
        \[ W^t \geq W^{t+1} \]

        C. If $z_t \neq x_t$ (if we make a mistake at time $t$), then at the beginning of round $t$ the total weight $I^{t-1}$ of the experts that make a mistake in round $t$ satisfies $I^{t-1} \geq \frac{W^{t-1}}{2}$.

        Then,
        \begin{equation*}
            \begin{split}
                W^t &= \frac{I^{t-1}}{2} + (W^{t-1} - I^{t-1}) = W^{t-1} - \frac{I^{t-1}}{2} \leq\\
                    &\leq W^{t-1} - \frac{W^{t-1}}{4} = \frac{3}{4} W^{t-1}
            \end{split}
        \end{equation*}

        D. After round $T$, if our algorithm has made $m$ mistakes, then
        \[ W^t \leq \left( \frac{3}{4} \right)^m \cdot W^0 = \left( \frac{3}{4} \right)^m \cdot n \]

        E. If $i^*$ is an expert that has made the fewest number of mistakes $m^*$, then
        \[ W^T = \sum_{i=1}^{m} w_i^T \geq w_{i^*}^T \]

        \begin{equation*}
            \begin{split}
                &2^{-m^*} = w_{i^*}^T \leq W^T \leq \left( \frac{3}{4} \right)^m \cdot n\\
                &m \log_2 \frac{3}{4} + \log_2 n \geq -m^*\\
                &-m \log_2 \frac{4}{3} + \log_2 n \geq -m^*\\
                &m^* + \log_2 n \geq m \log_2 \frac{4}{3}\\
                &m \leq \frac{1}{\log_2 \frac{4}{3}} (m^* + \log_2 n) \leq 2.41(m^* + \log_2 n)
            \end{split}
        \end{equation*}
    \end{proof}

    By changing the $\frac{1}{2}$ factor, we can get to a $(2 + \varepsilon)m^* + \frac{O(1)}{\varepsilon} \log n$ upper bound.


\subsection{Deterministic algorithms for the Expert problem}
    Let us consider the worst possibile deterministic algorithm, that says always the opposite of nature.

    Consider for instance Table~\ref{table:expert_example}.

    \begin{table}[h]
        \begin{tabular}{|l|c|c|c|r|c|}
        \hline
                                & $t=1$                          & $t=2$                          & $t=2$                          & \multicolumn{1}{c|}{$\cdots$} & $t = T$   \\ \hline
        $\text{Expert}_0$       & 0                              & 0                              & 0                              & $\cdots 0$                    & 0         \\ \hline
        $\text{Expert}_1$       & 1                              & 1                              & 1                              & $\cdots 1$                    & 1         \\ \hline
        Deterministic algorithm & $a_1$                          & $a_2$                          & $a_3$                          & $\dots a_{t-1}$               & $a_T$     \\ \hline
        Nature                  & \multicolumn{1}{l|}{$1 - a_1$} & \multicolumn{1}{l|}{$1 - a_2$} & \multicolumn{1}{l|}{$1 - a_3$} & $\dots 1 - a_{t-1}$           & $1 - a_T$ \\ \hline
        \end{tabular}
        \caption{Example of a bad deterministic algorithm}
        \label{table:expert_example}
    \end{table}

    Let the deterministic algorithm be any (fixed) deterministic algorithm.

    \begin{claim}
        At time $T$, the deterministic algorithm will have made $T$ mistakes
    \end{claim}

    \begin{proof}
        $z_t = a_t$, and $x_t = 1 - a_t \neq z_t$.    
    \end{proof}

    \begin{claim}
        At time $T$, $\text{Expert}_0$ or $\text{Expert}_1$ will have made at most $\frac{T}{2}$ mistakes.
    \end{claim}

    \begin{proof}
        If nature chooses a majority of $0$'s (if the deterministic algorithm chooses a majority of $1$'s) then $\text{Expert}_0$ will make no more than $\frac{T}{2}$ mistakes.

        Analogously, if nature chooses a majority of $1$'s (if the deterministic algorithm chooses a majority of $0$'s) then $\text{Expert}_1$ will make no more than $\frac{T}{2}$ mistakes.
    \end{proof}

    \begin{corollary}
        No deterministic algorithm can guarantee a number of mistakes $m \leq 2 m^*$.
    \end{corollary}


\subsection{Randomized algorithms}

    We will now consider randomized algorithms.
    The expert problem is an instance of a problem where randomization helps us a lot.
    In fact, it will let us "fool" nature; that is, nature won't be able to construct instances s.t. we are wrong w.p. $\frac{1}{2}$ (as it happend with deterministic algorithms).

    Consider Algorithm~\ref{alg:rand_weight_maj}.

    \begin{algorithm}
    \caption{$\text{Randomized Weighted Majority}_\varepsilon$}\label{alg:rand_weight_maj}
    \begin{algorithmic}%[1]
            \State assign weight $w_i = 1$, $\forall i \in [n]$

            \For{$t = 1, \dots, T$}
                \State choose randomly expert $i$ w.p. $\frac{w_i}{\sum_{j \in [n]} w_j}$
                \State suppose $i$ is sampled
                \State predict $z_t \overset{\Delta}{=} Y_{it}$
                \State see $x_t$ (and pay $1$€ is $x_t \neq z_t$)
                \For{$i = 1, \dots, n$}
                    \If{$Y_{it} \neq x_t$}
                        \State $w_i \gets (1 - \varepsilon) w_i$
                    \EndIf
                \EndFor
            \EndFor
    \end{algorithmic}
\end{algorithm}

    \begin{theorem}
        Fix $0 < \varepsilon < \frac{1}{2}$. The expected number of mistakes of Algorithm~\ref{alg:rand_weight_maj} is
        \[ \expected[m] \leq (1 + \varepsilon)m^* + \frac{1}{\varepsilon} \ln n \]
        Where $m^*$ is the number of mistakes of a best expert.
    \end{theorem}

    Above inequality is often written as
    \[ \expected[m] \leq m^* + \varepsilon \cdot m^* + \frac{1}{\varepsilon} \ln n \]
    Where $\varepsilon \cdot m^* + \frac{1}{\varepsilon} \ln n$ is called the "regret" of the algorithm.

    \begin{proof}
        Let $w_i^t$ be the weight of expert $i$ at the end of round $t$.

        Let $w_i^0 = 1$, $\forall i \in [n]$.

        Let $W^t = \sum_{i=1}^{m} w_i^t$ be our "potential".

        Let $I^t$ be the weighted fraction of experts that are wrong at time $t$,
        \[ I^t = \frac{\sum_{i : Y_{it} \neq x_t} w_i^{t-1}}{\sum_{i=1}^{n} w_i^{t-1}} = \frac{\sum_{i : Y_{it} \neq x_t} w_i^{t-1}}{W^{t-1}} \]

        Suppose $i^*$ is a best expert, and it makes $m^*$ mistakes.
        Then,
        \[ w_{i^*}^T = w_i^0 \cdot (1 - \varepsilon)^{m^*} = 1 \cdot (1 - \varepsilon)^{m^*} = (1 - \varepsilon)^{m^*} \]

        Thus,
        \[ W^T \geq w_{i^*}^T = (1 - \varepsilon)^{m^*} \]

        Moreover,
        \begin{equation*}
            \begin{split}
                \expected[m] &= \sum_{t=1}^{T} \prob[\text{\textsc{RWM} makes a mistake in round } t] =\\
                    &= \sum_{t=1}^{T} \prob[\text{\textsc{RWM} selects an incorrect expert in round } t] =\\
                    &= \sum_{t=1}^{T} \sum_{\substack{i \in [n] \\ Y_{it} \neq x_t}} \prob[\text{\textsc{RWM} selects expert } i \text{ in round } t] =\\
                    &= \sum_{t=1}^{T} \sum_{\substack{i \in [n] \\ Y_{it} \neq x_t}} \frac{w_i^{t-1}}{\sum_{j \in [n]} w_j^{t-1}} = \sum_{t=1}^{T} \sum_{\substack{i \in [n] \\ Y_{it} \neq x_t}} \frac{w_i^{t-1}}{\sum_{j \in [n]} W^{t-1}} =\\
                    &= \sum_{t=1}^{T} I^t
            \end{split}
        \end{equation*}

        Let $t \geq 0$, then
        \begin{equation*}
            \begin{split}
                W^{t+1} &= \sum_{i=1}^{n} w_i^t = \sum_{i=1}^{n}(w_i^t (1 - \varepsilon [Y_{i, t+1} \neq x_{t+1}])) =\\
                    &= \sum_{i=1}^{n} w_i^t - \varepsilon \sum_{i=1}^{n} (w_i^t [Y_{i, t+1} \neq x_{t+1}]) = \sum_{i=1}^{n} w_i^t - \varepsilon \sum_{\substack{i \in [n] \\ Y_{it} \neq x_t}} w_i^t =\\
                    &= \sum_{i=1}^{n} w_i^t - \varepsilon W^t \sum_{\substack{i \in [n] \\ Y_{it} \neq x_t}} \frac{w_i^t}{W^t} = W^t - \varepsilon W^t I^{t+1} = W^t (1 - \varepsilon I^{t+1})
            \end{split}
        \end{equation*}

        We then have
        \begin{equation*}
            \begin{split}
                &W^0 = \sum_{i=1}^{n} w_i^0 = n\\
                &W^1 = W^0 (1 - \varepsilon I^1) = n (1 - \varepsilon I^1)\\
                &W^2 = W^1 (1 - \varepsilon I^2) = n (1 - \varepsilon I^1)(1 - \varepsilon I^2)\\
                &\vdots\\
                &W^t = n \prod_{s=1}^t (1 - \varepsilon I^s)
            \end{split}
        \end{equation*}

        Then,
        \begin{equation*}
            \begin{split}
                &n \prod_{s=1}^{T} (1 - \varepsilon I^s) = W^T \geq (1 - \varepsilon)^{m^*}\\
                &\ln n - \varepsilon \sum_{s=1}^T I^S \geq \ln n + \sum_{s=1}^T \ln(1 - \varepsilon \cdot I^S) \geq m^* \ln (1 - \varepsilon)\\
                &\ln n \geq m^* \ln (1 - \varepsilon) + \varepsilon \sum_{s=1}^{T} I^s\\
                &\ln n \geq m^* \ln (1 - \varepsilon) + \varepsilon \cdot \expected[m]\\
                &\varepsilon \cdot \expected[m] \leq \ln n - m^* \ln (1 - \varepsilon)\\
                &\varepsilon \cdot \expected[m] \leq \ln n + m^* \ln \frac{1}{1 - \varepsilon}\\
            \end{split}
        \end{equation*}

        Thus,
        \begin{equation*}
            \begin{split}
                \expected[m] &\leq \frac{1}{\varepsilon} \ln n + \frac{m^*}{\varepsilon} \ln \frac{1}{1 - \varepsilon} \leq\text{\footnote{$\forall \varepsilon \in (0, \frac{1}{2})$, $\ln \frac{1}{1 - \varepsilon} \leq \varepsilon + \varepsilon^2$}}\\
                    &\leq \frac{1}{\varepsilon} \ln n + \frac{m^*}{\varepsilon} (\varepsilon + \varepsilon^2) =\\
                    &= (1 + \varepsilon) m^* + \frac{1}{\varepsilon} \ln n
            \end{split}
        \end{equation*}
    \end{proof}