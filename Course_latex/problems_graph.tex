\section{Problems on graphs}\label{sec:problems_graphs}

\def\cut{Cut(S, V \setminus S)}
\subsection{Max-Cut problem}\label{subsec:maxcut}
    Input: undirected graph $G(V,E)$

    Question: what is the bipartition $(S, V \setminus S)$ of $V$ that maximizes $\cardinality{\cut}$?

    Where a \textit{cut} is defined as follows:
    \[ \cut = \{ e \st e \in E \wedge e \cap S \neq \emptyset \wedge e \cap (V \setminus S) \neq \emptyset \} = \{ e \st e \in E \wedge \cardinality{e \cap S} = 1 \}\]

    This problem is NP-Hard, thus we don't know how to solve it efficiently.
    There exists a ($0.878\dots$)-approximation based on Semi-Definite Programming. We now see a simple $\frac{1}{2}$-approximation.

    \begin{algorithm}
    \caption{Approximation of Max-Cut}\label{alg:randomcut}
    \begin{algorithmic}%[1]
        \Procedure{Random-Cut}{$V,E$}
        \State~$S \gets \emptyset$
        \For{each $v \in V$}
            \State~flip an independent/fair coin $c_v$
            \If{$c_v$ is heads}
                \State~$S \gets S \cup \{ v \}$
            \EndIf
        \EndFor
        \State~\Return~$S$
        \EndProcedure
    \end{algorithmic}
\end{algorithm}


    \begin{theorem}\label{thm:randcut_approx}
        If $S^*$ is an optimal solution to Max-Cut on $G(V,E)$, and $S$ is the set returned by Random-Cut, then
        \[ \expected[ \cardinality{\cut} ] \geq \frac{1}{2} \cardinality{Cut(S^*, V \setminus S^*)} \]
    \end{theorem}

    \begin{lemma}\label{lemma:randcut_1}
        Let $S$ be the set returned by Random-Cut, then $\forall e \in E$, $\prob[ e \in \cut] = \frac{1}{2}$.
    \end{lemma}

    \begin{proof}
        By definition, $e \in \cut$ iff $\cardinality{S \cap e} = 1$.

        If $e = \{ v,w \}$, $e \in \cut$ iff $( (v \in S \wedge w \not\in S) \vee (v \not\in S \wedge w \in S) )$.

        Let $\xi_1 = "v \in S \wedge w \not\in S"$ and $\xi_2 = "v \not\in S \wedge w \in S"$.

        $\prob[\xi_1] = \prob[c_v = \text{Heads} \wedge c_w = \text{Tails}] = \text{(by independence of the coins) } \prob[c_v = \text{Heads}] \cdot \prob[c_w = \text{Tails}] = \frac{1}{2} \cdot \frac{1}{2} = \frac{1}{4}$

        With same reasoning, $\prob[\xi_2] = \prob[c_v = \text{Tails}] \cdot \prob[c_w = \text{Heads}] = \frac{1}{4}$.

        $\prob[e \in \cut] = \prob[\xi_1 \vee \xi_2] = \prob[\xi_1] + \prob[\xi_2] - \prob[\xi_1 \wedge \xi_2] = \frac{1}{4} + \frac{1}{4} + 0 = \frac{1}{2}$
    \end{proof}
    
    \begin{lemma}
        $\forall T \subseteq V$, $Cut(T, V \setminus T) \subseteq E$, $\cardinality{Cut(T, V \setminus T)} \leq E$.
    \end{lemma}

    \begin{proof}
        Trivial.
    \end{proof}

    \begin{corollary}\label{cor:randcut_3}
        Let $S$ be the set of nodes returned by Random-Cut, 
        \[ \expected[ \cardinality{\cut}] = \frac{\cardinality{E}}{2} \]
    \end{corollary}

    \begin{proof}
        By Lemma \ref{lemma:randcut_1}, $\forall e \in E$ $\prob[e \in \cut] = \frac{1}{2}$.

        $ \expected[\cardinality{\cut}] = \sum_{e \in E} \prob[e \in \cut] = \sum_{e \in E} \frac{1}{2} = \frac{\cardinality{E}}{2} $
    \end{proof}

    Recall Theorem \ref{thm:randcut_approx}.
    \begin{proof}
        The proof follows from Lemma \ref{lemma:randcut_1} and Corollary \ref{cor:randcut_3}.
    \end{proof}


    We now want to exploit a technique often used to improve the expected output of a randomized algorithm. We run the algorithm not once, but multiple times, and get the best result.

    \begin{algorithm}
    \caption{Repetition of Random-Cut}\label{alg:randomcut_rep}
    \begin{algorithmic}%[1]
        \For{$i=1$ to $t$}
            \State~run Random-Cut($V,E$) independently
            \State~let $S_i, V \setminus S_i$ be the resulting cut
            \State~let $C_i = \cardinality{\cut}$
        \EndFor
        \State \Return a largest cut
    \end{algorithmic}
\end{algorithm}
    
    Note that, at the end of the algorithm, we don't ask to return \textit{the} largest cut but \textit{a} largest cut, since it could happen that there are multiple cuts with the same cardinality.
    In fact, it can even happen, in general, that all the cuts found have the same cardinality.

    Now, for $i \in [t] = \{ 1, 2, \dots, t \}$, let us define $N_i = \cardinality{E} - C_i$ ($N_i \geq 0$). $N_i$ would be our error, the number of edges we didn't cut.

    Let $0 < \varepsilon < 1$.

    $\prob[N_i \geq (1+\varepsilon)\expected[N_i]] \overset{\text{(by \nameref{subsec:markov_ineq})}}{\leq} \dfrac{1}{1+\varepsilon} = \dfrac{1+\varepsilon}{1+\varepsilon} - \dfrac{\varepsilon}{1 + \varepsilon} = 1 - \dfrac{\varepsilon}{1 + \varepsilon} \leq 1 - \dfrac{\varepsilon}{2}$

    Note that $\cardinality{E} = 2 \expected[C_i]$.

    $\prob[N_i \geq (1+\varepsilon)\expected[N_i]] = \prob[\cardinality{E} - C_i \geq (1+\varepsilon)(\cardinality{E} - \expected[C_i])] = \prob[\cardinality{E} - (1 + \varepsilon) \cardinality{E} \geq C_i - (1 + \varepsilon) \expected[C_i]] = \prob[- \varepsilon \cardinality[E] \geq C_i - (1 + \varepsilon) \expected[C_i]] = \prob[-2 \varepsilon \expected[C_i] + (1+\varepsilon) \expected[C_i] \geq C_i] = \prob[C_i \leq (1-\varepsilon) \expected[C_i]]$

    $C_i \leq (1-\varepsilon) \expected[C_i]$ is our "bad event".

    $C_i > (1-\varepsilon) \expected[C_i]$ is our "good event". $\prob[C_i > (1-\varepsilon) \expected[C_i]] \geq \frac{\varepsilon}{2}$.

    Suppose we run for $t = \lceil\frac{2}{3} \ln \frac{1}{\delta} \rceil $ ($\delta$ will be the probability of error).

    $\prob[\forall i \in [t], C_i \leq (1-\varepsilon) \expected[C_i]] \overset{(\text{by ind.~of the } t \text{ runs})}{=} \prod_{i = 1}^t \prob[C_i \leq (1 - \varepsilon) \expected[C_i]] \leq \prod_{i = 1}^t (1 - \frac{\varepsilon}{2}) = (1 - \frac{\varepsilon}{2})^t \leq (e^{- \frac{\varepsilon}{2}})^t = e^{-\frac{\varepsilon}{2}t} \leq e^{-\frac{\varepsilon}{2} \frac{2}{\varepsilon} \ln \frac{1}{\delta} = e^{- \ln \frac{1}{\delta}} } = \delta$


\subsection{Vertex Cover problem}\label{subsec:vertcover}

    Input: undirected graph $G(V,E)$.

    Output: $S \subseteq V$, of smallest cardinalityt, s.t. $\forall e \in E$ $e \cap S \neq \emptyset$.

    In other words, we want a subset of the vertices s.t. \textit{every} edge of the graph touches at least one of the chosen vertices.

    It is a \textit{minimization} problem; note in fact that we could take every vertex (i.e. $S = V$) and this would clearly cover every edge.

    It is an NP-Complete problem.

    \begin{algorithm}
    \caption{Approximation to VC}\label{alg:maxmatching}
    \begin{algorithmic}%[1]
        \Procedure{MaximalMatching}{$V,E$}
        \State~$S \gets \emptyset$
        \While{$E \neq \emptyset$}
            \State~pick $e = \{u,v\} \in E$
            \State~$S \gets S \cup e$
            \State~remove all the edges incident on $v$ and $w$ from $E$
        \EndWhile
        \State\Return~$S$
        \EndProcedure
    \end{algorithmic}
\end{algorithm}

    The way we solve this problem is by actually solving another problem; in fact, we find a \textit{maximal matching}, that happens to be a VC.

    \begin{definition}[Matching]
        A matching of the graph $G(V,E)$ is a subset $A \subseteq E$ s.t. $\forall \{ e,e' \} \in \binom{A}{2}$, $e \cap e' = \emptyset$.
    \end{definition}

    \begin{definition}[Maximal matching]
        A maximal matching of $G(V,E)$ is a subset $A \subseteq E$ s.t.
        \begin{itemize}
            \item it is a matching
            \item $\forall e \in E \setminus A$, $A \cup \{e\}$ is not a matching
        \end{itemize}
    \end{definition}

    Note that there is a distinction between \textit{maximal} and \textit{maximum} matching.
    A maximum matching is a matching with the largest possible cardinality.
    A maximal matching is a matching s.t.~we can't add any more edges to it; so if we do so, the set is not a matching anymore.

    So it could happen that a matching is maximal, but there exists another matching with larger cardinality.

    \begin{lemma}\label{lemma:vc1}
        If $e_1, \dots, e_t$ are the edges selected by the MaximalMatching algorithms, then $\{ e_1, \dots, e_t \}$ is a maximal matching.
    \end{lemma}

    \begin{proof}
        For brevity, let us call $A = \{ e_1, \dots, e_t \}$.
        Suppose by contradiction that $A$ is not a maximal matching.

        So, either $A$ is not a matching at all, or is not maximal, so there exists an edge removed by the algorithm that should have been added to the set.

        When the algorithms picks an edge $e$, it removes all the edges connected to both endpoints of $e$. This ensures that $\forall e_1,e_2 \in A$, $e_1 \cap e_2 = \emptyset$.
        So we know that the set is indeed a matching, and now we have to check if it is maximal.

        If it's not maximal, then $\exists e \in E \setminus A$ that has been wrongfully removed by the algorithm.
        But, following the same reasoning used to show that $A$ is a matching, we know that $e$ shares some endpoints with at least one of the selected edges; i.e. $\exists e_A \in A$ s.t. $e \cap e_A \neq \emptyset$.

        So, adding any of the edges removed by the algorithm, would "break" the matching. Thus $A$ is also maximal.
    \end{proof}

    \begin{lemma}\label{lemma:vc2}
        If MaximalMatching selects $t$ edges, then $\cardinality{S} = 2t$.
    \end{lemma}

    \begin{proof}
        Trivial.
    \end{proof}

    \begin{lemma}\label{lemma:vc3}
        Let $A$ be any matching of $G(V,E)$. If $S$ is a VC of $G(V,E)$, then $\cardinality{S} \geq \cardinality{A}$.
    \end{lemma}

    \begin{proof}
        First note that a VC for $G(V,E)$ is also a VC for $G(V,B)$, $\forall B \subseteq E$.

        Now, pick any $A \subseteq E$. Since, by assumption, $S$ is a VC for $G(V,E)$, then $S$ must also cover each edge in $A$; this $S$ is a VC for $G(V,A)$.

        The graph $G(V,A)$ can only have nodes of degree at most $1$ (i.e.~either $0$ or $1$).
        In fact, suppose $A$ has a node of degree $>1$, say $2$; then $A$ would have two edges that share a node, thus it wouldn't be a matching.

        So, from the point of view of VC, any node in $G(V,A)$ can cover at most one edge, since $\forall v \in V$, $\deg_{G(V,A)}(v) \leq 1$.

        Thus, a set of, say, $k$ nodes, can cover $\leq k$ edges of $G(V,A)$. Now, $G(V,A)$ has $\cardinality{A}$ edges, thus each VC of $G(V,A)$ has to have at least $\cardinality{A}$ nodes.
    \end{proof}

    \begin{theorem}
        MaximalMatching returns a $2$-approximation to VC.\@
    \end{theorem}

    \begin{proof}
        Let $A$ be the maximal matching produced by the MaximalMatching procedure (the one called $\{ e_1, \dots, e_t \}$ in Lemma~\ref{lemma:vc1}).

        This solution contains $\cardinality{S} = 2 \cardinality{A}$ nodes (Lemma~\ref{lemma:vc2}).

        Since $A$ is a matching, if $S^*$ is an optimal solution (smallest solution), we know by Lemma~\ref{lemma:vc3} that $\cardinality{S^*} \geq \cardinality{A}$.

        $\cardinality{S^*} \geq \cardinality{A} = \frac{\cardinality{S}}{2} \rightarrow \cardinality{S} \leq 2 \cardinality{S^*}$.
    \end{proof}

    There are reasons to believe that this is the best possible approximation.
    In fact, VC is \textit{conjectured} to be NP-Hard to approximate to $2 - \varepsilon$ ($\forall \varepsilon > 0$ constant).

    Although, good news, VC has a good property that lets us "easily" (under some conditions) find a solution. It is \textbf{fixed parameter tractable}.
    If $G(V,E)$ has a VC of $k$ nodes, then an optimal VC of $G(V,E)$ can be found in time $n^c \cdot f(k)$; specifically, there's an algorithm that finds a VC in time $O(n^2 \cdot 2^k)$.

    So, if the optimal solution is small enough (i.e. $\log n$), then this search only takes polynomial time.

    The algorithm makes use of \nameref{subsubsec:induced_subgraph}.

    \begin{algorithm}
    \caption{Search of a VC}\label{alg:vc}
    \begin{algorithmic}%[1]
        \Procedure{VC}{$G(V,E), k$}
            \If{$E = \emptyset$}
                \State~\Return~true
            \EndIf
            \If{$k = 0$}
                \State~\Return~false
            \Else
                \State~fix $\{ u,v \} \in E$
                \If{VC($G[V \setminus \{u\}], k-1$)}
                    \State~\Return~true
                \EndIf
                \If{VC($G[V \setminus \{v\}], k-1$)}
                    \State~\Return~true
                \EndIf
            \EndIf
            \State~\Return~false
        \EndProcedure
    \end{algorithmic}
\end{algorithm}

    \begin{lemma}
        VC($G(V,E), k$) takes time $O(n^2 \cdot 2^k)$.
    \end{lemma}

    \begin{proof}
        By induction, running VC($G(V,E), l)$ causes at most $2^l - 1$ calls to the function VC.

        Base case ($l = 0$). Only the first call to VC($\cdot, 0$) will be generated. $2^{l-1} - 1 = 2^1 - 1 = 1$.

        Inductive step ($l+1$). Assume the claim holds for $l$. The number of calls generated by VC($\cdot, l+1$) is equal to no more than twice the numner of calls generated by VC($\cdot, l$)$+1$.
        The total number of calls, by induction, is then $\leq 1 + 2(2^{l+1}-1) = 1 + 2^{l+2} - 2 = 2^{l+2} - 1$.

        Each single casll takes time $O(n^2)$; there a few checkes that can be done in constant time, we have to fix an edge, and we have to build the induced subgraph (which take $O(n^2)$).
        Note that, fixing an edge might take $O(n^2)$; worst case we use an adjacency matrix and we visit it until we find a fixable edge.

        Thus, the claim follows.
    \end{proof}

    \begin{lemma}
        VC($G(V,E), k$) == true iff $G(V,E)$ has a VC of size $k$.
    \end{lemma}

    \begin{exercise}
        The proof of above lemma is left as an exercise
    \end{exercise}