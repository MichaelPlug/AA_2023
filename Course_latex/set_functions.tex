\section{Set functions}
    Given $S \subseteq [n]$, a set function is a function $f : 2^{[n]} \rightarrow \mathbb{R}$, where $f(S)$ is the value of $f$ at $S$.

    Set functions can be divided in three kinds.

    \textbf{Submodular functions}, $\forall S,T \subseteq [n]$:
    \[ f(S) + f(T) \geq f(S \cup T) + f(S \cap T) \]

    \textbf{Modular functions}, $\forall S,T \subseteq [n]$:
    \[ f(S) + f(T) = f(S \cup T) + f(S \cap T) \]
    Also, a function is modular iff $\exists$ weights $w_1, \dots, w_n, z$ s.t. $f(S) = z + \sum_{i \in S} w_i$.

    \textbf{Supermodular functions}, $\forall S,T \subseteq [n]$:
    \[ f(S) + f(T) \leq f(S \cup T) + f(S \cap T) \]

    Let us consider the following set function.
    Let $C = \{ A_1, \dots, A_n \}$, where $A_i \subseteq [t]$.
    Let our "covering" function be
    \[ f(S) = \cardinality{\bigcup_{j \in S} A_j} \]
    for $S \subseteq [n]$.

    \begin{theorem}
        The function $f$, as just defined, is submodular.
    \end{theorem}

    \begin{proof}
        Observe that $f(S) = \sum_{i \in [t]} [\exists j \in S : i \in A_j] = \sum_{i=1}^{t} f_i(S)$, where
        \begin{equation}
            f_i(S) =
            \begin{cases}
                1 & \text{if } \exists j \in S \text{ s.t. } i \in A_j\\
                0 & \text{o/w}
            \end{cases}
        \end{equation}

        We claim that, $\forall i \in [t]$, $f_i$ is submodular.
        To prove it, we need to prove that $f_i(S) + f_i(T) \geq f_i(S \cup T) + f_i(S \cap T)$, $\forall S,T \subseteq [n]$.

        Since the range of $f_i$ is (a subset of) $\B$, if $f_i(S \cup T) = f_i(S \cap T) = 0$, then the inequality holds trivially.

        Moreover, $f_i$ is monotone increasing: if $S \subseteq T$, then $f(S) \leq f(T)$.
        If $f_i(S) = 1$, then $\exists j \in S$ s.t. $i \in A_j$. Since $j \in S \subseteq T$, then it must be that $f_i(T) = 1$.

        Thus we only need to consider two cases:
        \begin{itemize}
            \item[a.] $f_i(S \cup T) = f_i(S \cap T) = 1$
            \item[b.] $f_i(S \cap T) = 0$ and $f_i(S \cup T) = 1$
        \end{itemize}

        If a. holds, then $\exists j \in S \cap T$ s.t. $i \in A_j$, but then $j \in S$ and $j \in T$, thus $f_i(S) = f_i(T) = 1$

        if b. holds, then $\exists j \in S \cup T$ s.t. $i \in A_j$, thus $f_i(S \cup T) = 1$. Either $j \in S$ or $j \in T$, and not both. Thus $f_i(S) + f_i(T) = 1$.

        Thus, we have proved that $\forall i \in [t]$, $f_i$ is submodular.
        To finish the proof we have to show that the sum of submodular functions is submodular.

        $f(S) = \sum_{i=1}^{t} f_i(S)$·

        Let $S,T \subseteq [n]$, then
        \begin{equation*}
            \begin{split}
                f(S) + f(T) &= \sum_{i=1}^{t} (f_i(S) + f_i(T)) \overset{f_i \text{ submodular}}{\geq}\\
                    &\geq \sum_{i=1}^{t}(f_i(S \cup T) + f_i(S \cap T)) =\\
                    &= \sum_{i=1}^{t} f_i(S \cup T) + \sum_{i=1}^{t} f_i (S \cap T) =\\
                    &= f(S \cup T) + f(S \cap T)
            \end{split}
        \end{equation*}
    \end{proof}

    We now study Max-cover as being a special case of submodular functions.

    Submodular functions are also studied in economics, where they are given a different name.
    \begin{definition}[Diminishing return]
        Given $S \subseteq [n]$ and $x \in [n] \setminus S$, we define the return of $x$ at $S$ to be
        \[ \Delta_f (x \st S) = f(S \cup \{ x \}) - f(S) \]
        That is, how much do I gain by adding $x$ to $S$.
    \end{definition}

    A function $f$ has the diminishing return property if, $\forall S \subseteq T$, $\forall x \in [n] \setminus T$:
    \[ \Delta_f(x \given S) \geq \Delta_f(x \given T) \]

    \begin{lemma}
        If $f$ is submodular then $f$ has the diminishing return property.
    \end{lemma}

    \begin{proof}
        We'd like to prove that $\forall A \subseteq B$, $\forall x \in [n] \setminus B$, $\Delta_f(x \given A) \geq \Delta_f(x \given B)$.

        Let us define $S = A \cup \{x\}$, $T = B$.

        By submodularity of $f$, $f(S) + f(T) \geq f(S \cup T) + f(S \cap T)$.
        Thus $f(A \cup \{x\}) + f(B) \geq f(A \cup \{x\} \cup B) + f((A \cup \{x\}) \cap B)$.

        $f(A \cup \{x\}) + f(B) \geq f(B \cup \{x\}) + f(A)$

        $f(A \cup \{x\}) - f(A) \geq f(B \cup \{x\}) - f(B)$

        Then, $\Delta_f(x \given A) \geq \Delta_f(x \given B)$.
    \end{proof}

    \begin{exercise}
        The converse of previous lemma also holds.
        Prove that if $f$ has the diminishing return property, then $f$ is submodular.
    \end{exercise}

    We aim to maximize a submodular $f$ under a "cardinality constraint".
    In Max-cover, we wanted to maximize $f(S)$ under $\cardinality{S} = k$.

    In algorithm~\ref{alg:greedy_submodular}

    \begin{algorithm}
    \caption{Submodular functions framework for Max-cover}\label{alg:greedy_submodular}
    \begin{algorithmic}%[1]
        \Procedure{$\textsc{Greedy}_f$}{$X,k$}
            \State~$S_0 \gets \emptyset$

            \For{$i = 0,1, \dots, k-1$}
                \State~let $x_{i+1} \in X$ be a maximizer of $f(S_i \cup \{x_{i+1}\})$  %\Comment{equiv. to max of $\Delta_f(x_{i+1} \given S_i)}
                \State~$S_{i+1} \gets S_i \cup \{x_{i+1}\}$
            \EndFor
            
            \Return~$S_k$
        \EndProcedure
    \end{algorithmic}
\end{algorithm}

    \begin{theorem}
        If $f$ is submodular, monotone and non-negative, then $\textsc{Greedy}_f(X,k)$ returns a $(1 - \frac{1}{e})$-approximation to $\max_{S \in \binom{X}{k}} f(S)$.
    \end{theorem}

    \begin{proof}
        Let $S^* = \{ X_1^*, \dots, X_k^* \}$ be an optimal solution.
        
        Then $f(S^*) = \max_{S \in \binom{X}{k}} f(S)$.

        $\forall i = 0, \dots, k-1$:
        %\begin{multline*}
        %    f(S^*) \overset{\text{monotonicity}}{\leq} f(S^* \cup S_i) =\\
        %    f(\{ X_1^*, \dots, X_k^* \} \cup S_i) - f(\{ X_1^*, \dots, X_{k-1}^*\} \cup S_i) + f(\{ X_1^*, \dots, X_{k-1}^* \} \cup S_i) -\\
        %    f(\{ X_1^*, \dots, X_{k-2}^* \} \cup S_i)  + f(\{ X_1^*, \dots, X_{k-2}^* \} \cup S_i)  - \cdots - f(\{ X_1^*, X_2^*\} \cup S_i) + \\
        %    f(\{ X_1^*, X_2^*\} \cup S_i)  - f({X_1^*} \cup S_i) + f({X_1^*} \cup S_i) - f(S_i) + f(S_i) =\\
        %    \Delta_f(X_{k}^* \given S_i \cup \{ X_1^*, \dots, X_{k-1}^* \}) +\\
        %    \Delta_f(X_{k-1}^* \given S_i \cup \{ X_1^*, \dots, X_{k-2}^* \}) +\\
        %    \vdots\\
        %    \Delta_f(X_{2}^* \given S_i \cup \{ X_1^*\}) +\\
        %    \Delta_f(X_{1}^* \given S_i) +\\
        %    f(S_i) =\\
        %    f(S_i) + \sum_{j=1}^{k} \Delta_f(X_j^* \given S_i \cup \{ X_1^*, \dots, X_{j-1}^* \}) \overset{\text{diminish. return}}{\leq} f(S_i) + \sum_{j=1}^{k} \Delta_f(X_j^* \given S_i) \overset{\text{greedy choice}}{\leq}\\
        %    f(S_i) + \sum_{j=1}^{k} \Delta_f(X_{i+1} \given S_i) = f(S_i) + k \cdot \Delta_f(X_{i+1} \given S_i) = f(S_i) + k (f(S_i \cup \{ X_{i+1} \}) - f(S_i))
        %\end{multline*}

        \begin{equation*}
            \begin{split}
                f(S^*) \overset{\text{monotonicity}}{\leq} & f(S^* \cup S_i) =\\
                    &= f(\{ X_1^*, \dots, X_k^* \} \cup S_i) - f(\{ X_1^*, \dots, X_{k-1}^*\} \cup S_i) + f(\{ X_1^*, \dots, X_{k-1}^* \} \cup S_i) -\\
                    &- f(\{ X_1^*, \dots, X_{k-2}^* \} \cup S_i)  + f(\{ X_1^*, \dots, X_{k-2}^* \} \cup S_i)  - \cdots - f(\{ X_1^*, X_2^*\} \cup S_i) + \\
                    &+ f(\{ X_1^*, X_2^*\} \cup S_i)  - f({X_1^*} \cup S_i) + f({X_1^*} \cup S_i) - f(S_i) + f(S_i) =\\
                    &= \Delta_f(X_{k}^* \given S_i \cup \{ X_1^*, \dots, X_{k-1}^* \}) +\\
                    &+ \Delta_f(X_{k-1}^* \given S_i \cup \{ X_1^*, \dots, X_{k-2}^* \}) +\\
                    & \vdots\\
                    &+ \Delta_f(X_{2}^* \given S_i \cup \{ X_1^*\}) +\\
                    &+ \Delta_f(X_{1}^* \given S_i) +\\
                    &+ f(S_i) =\\
                    &= f(S_i) + \sum_{j=1}^{k} \Delta_f(X_j^* \given S_i \cup \{ X_1^*, \dots, X_{j-1}^* \}) \overset{\text{diminish. return}}{\leq}\\
                    &\leq f(S_i) + \sum_{j=1}^{k} \Delta_f(X_j^* \given S_i) \overset{\text{greedy choice}}{\leq}\\
                    &\leq f(S_i) + \sum_{j=1}^{k} \Delta_f(X_{i+1} \given S_i) = f(S_i) + k \cdot \Delta_f(X_{i+1} \given S_i) =\\
                    &= f(S_i) + k (f(S_i \cup \{ X_{i+1} \}) - f(S_i))
            \end{split}
        \end{equation*}

        Then, $f(S^*) \leq f(S_i) + k (f(S_i \cup \{ X_{i+1} \}) - f(S_i))$.

        Let us define $\delta_i = f(S^*) - f(S_i)$.

        $f(S^* - f(S_i)) \leq k (f(S_{i+1}) - f(S_i))$

        $\delta_i \leq k (-f(S^*) + f(S_{i+1}) - f(S_i) + f(S^*))$

        $\delta_i \leq k (\delta_i - \delta_{i+1})$

        $k \delta_{i+1} \leq (k-1) \delta$

        Then, $\forall i = 0, \dots, k-1$, $\delta_{i+1} \leq (1 - \frac{1}{k}) \delta_i$.

        Finally,
        %$\delta_0 = f(S^*) - f(S_0) = f(S^*) - f(\emptyset) \overset{\text{non-neg.}}{\leq} f(S^*)$
        %$\delta_1 \leq (1 - \frac{1}{k}) \delta_0 \leq (1 - \frac{1}{k}) f(S^*)$
        %$\delta_2 \leq (1 - \frac{1}{k}) \delta_1 \leq (1 - \frac{1}{k})^2 f(S^*)$
        %$\vdots$
        %$\delta_k \leq f(S^*) - f(S_k) \leq (1 - \frac{1}{k})^k f(S^*) \leq \frac{1}{e} f(S^*) \rightarrow f(S_k) \geq (1 - \frac{1}{e}) f(S^*)$

        \begin{equation*}
            \begin{split}
                &\delta_0 = f(S^*) - f(S_0) = f(S^*) - f(\emptyset) \overset{\text{non-neg.}}{\leq} f(S^*)\\
                & \delta_1 \leq (1 - \frac{1}{k}) \delta_0 \leq (1 - \frac{1}{k}) f(S^*)\\
                & \delta_2 \leq (1 - \frac{1}{k}) \delta_1 \leq (1 - \frac{1}{k})^2 f(S^*)\\
                & \vdots\\
                & \delta_k \leq f(S^*) - f(S_k) \leq (1 - \frac{1}{k})^k f(S^*) \leq \frac{1}{e} f(S^*) 
            \end{split}
        \end{equation*}

        Then, $f(S_k) \geq (1 - \frac{1}{e}) f(S^*)$.
    \end{proof}

    Let us now see some problems on submodular functions.
    Let $f : 2^{[n]} \rightarrow \mathbb{R}$ be submodular.

    Problem: assuming that $f$ is monotone increasing and submodular, then
    \[ \max_{S \subseteq [n] : \cardinality{S} \leq k} f(S) \]
    Can be approximated to $1 - \frac{1}{e}$.

    Problem: assuming that $f$ is submodular, then
    \[ \min_{S \subseteq [n]} f(S) \]
    Can be solved exactly in polytime.

    Problem: assuming that $f$ is submodular, then
    \[ \max_{S \subseteq [n]} f(S) \]
    Is NP-Hard.